%!TEX root =  main.tex
%!TEX encoding = UTF-8 Unicode
\chapter{協調フィルタリング:ハイブリッド}
\label{sec:cfsummary}

協調フィルタリングはメモリとモデルベースとに分類できると述べたが,これらの中間的な手法も存在する.

personality diagnosis法\cite{uai:00:01}は,メモリベースのように利用者間の類似度を用いる.
\index{personality diagnosis method}
そして,この類似度を重みとした,各利用者の評価値の重み付平均を推定評価値とする.
これだけであれば,メモリベース法だが,類似度計算にモデルベースの要素がある.
各利用者に個人モデルを作成し,活動利用者と標本利用者の個人モデルが一致する確率を類似度としている.
%@@@ モデルベース法の最後に移動

クラスタリングをしたあとに,メモリベース法を適用する方法は,メモリベース法に,モデルベースの要素を加えた手法と解釈できる.
クラスタリングする対象が,利用者集合のもの\cite{sigir:05:01}と,アイテム集合のものと\cite{misc:090}とがある.
前者の方法では,アイテムへの嗜好パターンが類似した利用者のクラスタを生成する.
活動利用者と嗜好パターンが近いクラスタを見つけ,そのクラスタ内の利用者を対象に利用者間型メモリベース法を適用する.
利用者間型メモリベース法で近傍利用者を利用した場合と同様に,推薦の高速化と,場合によっては予測精度の向上が見込めるが,適応性の
面では不利になる.
後者のアイテム集合をクラスタリングする方法では,
\ref{sec:item-item}のように,$S$の列ベクトルの類似性に基づいてクラスタを作る.
その後,各クラスタごとに個別にメモリベース手法を適用する.
これは,特定のアイテムカテゴリ内に推薦対象を限定すると,そのアイテム群に特に関心のある利用者に対する予測精度が向上するとの考えに基づくが,実験ではその有効性は確認できなかったと報告している.
%@@@ モデルベースのクラスタリングに入れる

%Google News\cite{www:07:01}では,モデルベースとメモリベースの結果を\ref{sec:combmethod}のメタ推薦のように,重み付線形和でスコアを統合して最終結果を得ている.
