%!TEX root =  main.tex
%!TEX encoding = UTF-8 Unicode
\chapter{協調フィルタリング}
\label{chap:cf}

ここでは,協調フィルタリングによる嗜好の予測について述べる.
ここでいう予測とは,活動利用者はまだ知らないが,他の標本利用者は知っているアイテムについて,活動利用者の関心の有無や,評価値を推定することである.
このように,与えられたデータの中から,規則性を見つけ出し,その規則性に基づいて予測する問題は,機械学習や統計的予測の枠組みによって解く.
だが,万能な予測手法は理論的にありえないことは,ノーフリーランチ定理として知られている.
よって,\ref{chap:design}章で述べた,利用者数やアイテム数などのデータの特性,望ましい推薦が備えるべき性質を考慮してモデルやアルゴリズムを選択しなくてはならない.
もしこれらが適切でなければ,データがいくらあっても予測精度が向上することはないので,この選択は重要である.
これらの選択に関連した,モデルやアルゴリズムの特徴に留意して述べるので,参考にされたい.

\section{メモリベース法とモデルベース法}
\label{sec:memory-model}

個々のアルゴリズムについて述べる前に,協調フィルタリング手法の分類と,それぞれのグループの長所と短所を述べる.
推薦候補の予測手法は\term{メモリベース法}{memory-based method}(事例ベース法 (instance-based method)ともいう)と
\term{モデルベース法}{model-based method} に分けられる\cite{uai:98:01}.
メモリベース法では,推薦システムが利用される以前には何もせず,ただ利用者DBを保持している.
\index{利用者データベース}\index{user database}
そして,推薦をするときに,利用者DB中の嗜好データそのものと,活動利用者の嗜好データを併せて予測をする.
もう一方のモデルベース法では,推薦システムが利用される以前に,あらかじめモデルを構築する.
このモデルは,「佐藤さんが好むものは,鈴木さんも好むことが多い」といった,利用者とアイテムの嗜好についての規則性を表す.
推薦をするときに,利用者DBは用いずに,このモデルと活動利用者の嗜好データとに基づいて予測する.
すなわち,事前にモデルを構築するかどうかという違いが重要である.

\begin{table}
\centering
\caption{メモリベース法とモデルベース法の比較}
\label{tab:memory-model}
\begin{tabular}{l@{\qquad}>{\centering}p{8zw}>{\centering}p{8zw}p{0pt}}\toprule
 & メモリベース法 & モデルベース法 & \\\midrule
推薦時間 & ×\,:\,遅い & ○\,:\,速い & \\
適応性   & ○\,:\,あり & ×\,:\,なし & \\
\bottomrule
\end{tabular}
\end{table}

これら二つの手法の大まかな長所と短所を表\ref{tab:memory-model}にまとめておく.
推薦時間についてだが,メモリベース法は一般に遅い.
これは,利用者DBには多数の標本利用者や推薦対象が登録されており,これら多数の項目を推薦の度に調べ直すのに時間がかかるためである.
一方のモデルベース法では,夜間などシステムが利用されない間や,毎月や毎週など定期的にモデルをあらかじめ構築・更新しておく.
すると,この時間は推薦の時間からは除外できる.
また,モデルの規模は,利用者DBのそれと比べて小さいので,活動利用者に速く推薦することができる.

表\ref{tab:memory-model}の適応性とは,標本利用者や推薦対象が,削除されたり追加されたりしても,適切な推薦ができるかということである.
このような削除や追加をすると,モデルベース法では,利用者や推薦対象間の規則性に変化が生じるため,モデルを再構築する必要が生じる.
だが,モデルの構築には時間がかかるので,頻繁に行うことは難しい.
そのため,適応性に関してはモデルベース法は不利になる.
一方,メモリベース法では,モデルの構築は行わないためこのような問題は生じない.

%@@@ 協調フィルタリングの形式的な定式化についてまとめる:明示的・暗黙的,ch-memorybased の一部を移してくる
