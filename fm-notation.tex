%!TEX root =  main.tex
%!TEX encoding = UTF-8 Unicode
\chapter*{数式の表記}
\label{sec:notation}

スカラーの変数はイタリック体 $x$ で,一部の確立変数は大文字のイタリック体 $X$,ベクトルは小文字ボールド体 $\bfx$で,行列は大文字ボールド体 $\bfX$で表記する.
実数などの特殊なものを除き,集合にはカリグラフィック体 $\calD$ を用いる.

\begin{center}
\small\renewcommand{\arraystretch}{0.8}
\begin{tabular}{@{}>{\centering}p{0.07\fullwidth}@{\hspace{0.01\fullwidth}}p{0.40\fullwidth}@{\hspace{0.04\fullwidth}}>{\centering}p{0.07\fullwidth}@{\hspace{0.01\fullwidth}}p{0.40\textwidth}@{}}\toprule
\textbf{表記} & \textbf{内容} & \textbf{表記} & \textbf{内容} \\ \cmidrule(r){1-2}\cmidrule{3-4}
$x$ & 特定の利用者を表す & $y$ & 特定のアイテムを表す \\
$X$ & 利用者を表す確率変数 & $Y$ & アイテムを表す確率変数 \\
$\bfx$ & 利用者をまとめたベクトル & $\bfy$ & アイテムをまとめたベクトル \\
$n$ & 利用者数 & $m$ & アイテム数 \\
$\calX$ & 利用者集合 $\{1,\ldots,n\}$ & $\calY$ & アイテム集合 $\{1,\ldots,m\}$ \\
$\calX_y$ & アイテム$y$を評価した利用者の集合 & $\calY_x$ & 利用者xが評価したアイテムの集合 \\
$a$ & 活動利用者を表す & $r_{xy}$ & 利用者$x$のアイテム$y$への評価値 \\
$\bar{r}_x$ & 利用者$x$による評価値の平均 & $\tilde{r}_y$ & アイテム$y$への評価値の平均 \\
$\bfR$ & 評価値行列 & $\calR$ & 評価値集合(5段階評価なら $\{1,\ldots,5\}$) \\
$\bfr$ & 評価値をまとめたベクトル & $z$ & 潜在因子 \\
$\bfz$ & 潜在因子のベクトル & $K$ & 潜在因子の数・次元数 \\
$\calD$ & データ集合 & $N$ & 訓練データ数 \\
$\bfU$ & 利用者潜在因子行列 & $\bfV$ & アイテム潜在因子行列 \\
$\bfu_{x}$ & 利用者$x$の潜在因子ベクトル & $\bfv_{y}$ & アイテム$y$の潜在因子ベクトル \\
$y^{(t)}$ & 時刻$t$での値 & $\ang{Y^{(t)}}$ & アイテムの時系列 \\
$\theta\,\bftheta\,\bfTheta$ & パラメータを一般に表す & $\sig()$ & シグモイド関数 \\
$\Dom()$ & 変数$x$の定義域 & $\nan$ & 欠損値 \\
\bottomrule
\end{tabular}
\end{center}

スカラー関数 $f(x)$ に対して,その引数をベクトルとする表記 $f(\bfx)$ は,ベクトル $\bfx$ の各要素を関数 $f$ に適用して得られるベクトルを表す.

確率変数 $X$ が離散の場合の確率質量関数も,連続値の場合の確率密度関数も特に区別することなく $\Pr(X)$ と表記する.

$\expect_{p(X)}[f(X)]$ は,分布 $p(X)$ についての次の期待値を表す:
\[
\begin{array}{l@{\text{ --- }}l}
\sum_{x\in\Dom(X)} f(x) \Pr(x) & \text{$X$が離散の場合}\\
\int_{x\in\Dom(X)}  f(x) \Pr(x) dx & \text{$X$が連続の場合}
\end{array}
\]
なお,$p(X)$を省略した場合は,関数$f$の全ての確率変数の同時分布に関する期待値を表す.例えば,$\expect[f(X,Y)]$ は,$\expect_{p(X,Y)}[f(X,Y)]$ の意味である.
