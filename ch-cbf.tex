%!TEX root =  main.tex
%!TEX encoding = UTF-8 Unicode
\chapter{内容ベースフィルタリング}
\label{chap:cbf}

\ref{chap:process}章で述べたように,内容ベースフィルタリングを,やや広義にとらえ,アイテムや利用者自身の特徴に基づく推薦手法とした.
この内容ベースフィルタリングについて,次の三つの観点から述べる.
\begin{description}
 \item[特徴の種類] アイテムの特徴,個人属性特徴,コンテキストの特徴
 \item[入力の形式] 嗜好データと検索質問
 \item[推薦規則の獲得] 学習よる獲得と人手による定義
\end{description}
これらのうち,特徴の種類については\ref{sec:featuredata}節で述べたので,残り二つについて述べる.

%@@@ 知識ベースを導入したら合わせて変更する

\section{入力の形式}
\label{sec:cbfinput}

入力の形式には嗜好データと検索質問とがある.
嗜好データは,利用者の好の度合いを示したデータで,間接指定型内容ベースフィルタリング\cite{misc:091,tjsai:05:05,trjsai:06:01}で用いられる.
個々のデータの推薦への寄与は小さいので,継続的にデータを集積する必要がある.
そのため,永続的個人化(\ref{sec:plevel}節)のために主に利用する.

もう一方の検索質問とは,\ref{chap:process}章の冒頭で述べたような,アイテムの特徴に対する制約であり,直接指定型の内容ベース
\cite{ec:024,ijcai:03:04,jair:04:01,ieeem:07:06}で利用される.
特に,候補アイテムがテキストで,検索質問に索引語が利用される場合には情報検索\cite{jb:012:00}そのものとなる.
こうしたシステムは対話的に動作し,徐々に制約を強くして候補を絞り込むものが多い(\ref{sec:interactive}節).
このような明示的な検索質問の他に,暗黙的なものもある.
例えば,論文の推薦システムであるCiteSeer\cite{ieeem:99:02}では,現在閲覧している論文を検索質問とみなし,その論文と本文テキストや引用文献リストが類似している文献を推薦する.
検索質問は,継続的に蓄積しなくても利用者の嗜好に関する情報を得られるので,一時的個人化(\ref{sec:plevel}節)のために主に利用する.

%対話的なシステムでは,両方の種類の入力が使われる場合もある.
%例えば,最初に検索質問を入力し,制約を満たす推薦リストを提示する.
%その後,利用者の情報要求に適合するものを選択させることで,利用者から嗜好データを獲得する.
%その後は,この嗜好データを利用者適合フィードバック(user relevance feedback)として,情報検索の技術を用いて推薦リスト中の順位付けを洗練する.

\section{推薦規則の獲得}
\label{sec:cbfrule}

推薦するアイテムを決定する規則は,機械学習によって獲得する場合と,人手によって定義する場合がある.
学習による獲得は,永続的な推薦を嗜好データを用いて行う場合に採用する場合が多い.
機械学習問題としては,\term{クラス分類}{classification}問題\cite{eb:053:00,jpublist:077x,jb:033:00,jb:035:00,trieice:06:04}
に該当する問題である.
嗜好データの好き・嫌いをクラスとみなす.
一方,特徴ベクトルは,嗜好データを獲得したときのアイテム,個人属性,コンテキストの特徴で構成する.
これらのクラスと特徴ベクトルの対を訓練事例とし,機械学習アルゴリズムを適用すれば,アイテム,個人属性,コンテキストの特徴から,利用者のアイテムへの嗜好を予測できるようになる.
嗜好データを順序付カテゴリとみなす場合は,順序回帰問題\cite{jb:025:00}としてとらえることもできる.
検索質問入力を採用した場合にも機械学習によって推薦規則を洗練する方法もある\cite{jair:04:01}.
この方法では,検索質問で頻繁に利用される特徴や属性値について,類似度判定の際の重みを増やして,将来の検索で重視するようにしている.

もう一方の人手による定義とは,利用者が与えた効用関数,IF-THEN型ルール,または類似度関数などに基づいて,推薦するアイテムを決める方法で
ある.文献\cite{ej:048}では,効用ベース(utility-based)と知識ベース(knowledge-based)とに分類されている方法である.
入力が検索質問の場合には,類似度関数を用いて検索質問の近傍を推薦する手法が多用されされている.
こうした規則は,人間がヒューリスティックに作成するため,複雑で予測精度の高いものを作るのは難しい.
だが,純粋に手作業で定義するのではなく,学習により獲得した評価値をヒューリスティックな規則で修正することは比較的容易で,有用である.
例えば,特売品の予測評価値を増加させたり,在庫がない商品の予測評価値を下げたりといったように,システム管理者側の意図を反映させる場合などである.
