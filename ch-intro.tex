%!TEX root =  main.tex
%!TEX encoding = UTF-8 Unicode
\chapter{推薦システム}
\label{sec:intro}

「\term{推薦システム}{recommender system}」とは,利用者にとって有用と思われる対象,情報,または商品などを選び出し,それらを利用者の目的に合わせた形で提示するシステムである.

最初に,この推薦システムが必要になった背景について述べよう.
第一に,大量の情報が発信されるようになったことがある.
これは,情報化技術の進展により,個人・団体が容易かつ低コストで発信できるようになったためである.
第二の理由は,これら大量の情報の蓄積や流通が容易になり,誰もが大量の情報を得ることができるようになったことである.
これも計算機の記憶媒体の大規模化や,通信の高速化によるものである.
以上のことから,大量に発信された情報を,だれもが大量に取得できる状況が生じた.
しかし,欲しい情報が何か分からない(例:統計資料として公開されているがその名前が分からない)とか,探している情報を見つけ出せない(例:類似した資料が大量にあり目的のものが埋もれてしまう)といった理由により,情報を参照できる状態にあるにもかかわらず,それを利用できないという状況が生じた.
この状況を「\term{情報過多}{information overload}~\cite{misc:009}
\footnote{情報爆発 (information explosion) や情報洪水 (information overflow) ともいう.}
という.
この状況を打破するため,利用者にとって有用な情報を見つけ出す推薦システムが考案された.

この推薦システムは,広い立場からみれば,情報検索や情報フィルタリング技術の一つと見なせる.そのため,初期の推薦システムは,これらの技術が基盤となっていた.
\ref{sec:cfcbf}で述べるように,このシステムの実現手法には協調フィルタリングと内容ベースフィルタリングがある.
だが,協調フィルタリングという語の方が推薦システムより古く,1992年に文献 \cite{macm:92:01} にて使われた.
だが,これは現在のような協調フィルタリングではなく,他人が手動で行った推薦を検索できる協調作業支援のシステムであった.
この過程を自動化した協調フィルタリングが1994年のGroupLens\cite{cscw:94:01}やRingo \cite{sigchi:95:02}であり,現在の推薦システムの基礎となった.
一方の内容ベースフィルタリングの技術は,従来からある情報フィルタリングとみなせ,また,事例ベース推論の応用としても研究されてきた.
そのうち,推薦システムとして独自の側面が強くなり,協調フィルタリングに対して内容ベースフィルタリングと呼ばれるようになった.
1996年には,専門のワークショップも開催されるほどに研究が活発化した.
1997年にはACM Communications誌で特集\cite{macm:97:01} が掲載され,この種のシステムの呼び名として「Recommender System」が定着した.
このころから,NetPerceptions や Firefly などの企業によってシステムの商業化が始まった.
現在では,Webを通じた各種サービスの機能で活用されたり\cite{ieeem:99:02,ieeem:03:01,www:07:01},セットトップボックスなど
の機器に組み込まれたり\cite{kdd:04:11}している.
その後,物理的な店舗面積に商品数が制限されない電子商取引の発展や,大量の画一的な商品から,少量多品種への消費傾向の変化にに伴って,その重要性も広く認識されるようになった.
それを象徴する Amazon.com CEOのJeff Bezosの発言を引用しておこう\cite{dmkd:01:01}.
\footnote{J.~Bezosは,この発言を幾つかの講演で行った.ここでは,300万人と書いたが,そのときどきの顧客数に応じて,この数字は変えて用いられた.} % J.Riedlのメールより
\begin{center}
If I have 3 million customers on the Web, I should have 3 million stores on the Web\\
Webに3百万人の顧客がいるなら,3百万のWebストアを用意すべきだ
\end{center}
現在では,推薦システムは多方面で利用され,研究も継続的に行われ,多様な方法が目的に応じて考案されている.
これらの推薦システムの設計指針やアルゴリズムについてまとめる.
